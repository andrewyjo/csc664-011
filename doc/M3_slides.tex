% Created 2022-11-27 Sun 23:42
% Intended LaTeX compiler: pdflatex
\documentclass[11pt]{article}
\usepackage[utf8]{inputenc}
\usepackage[T1]{fontenc}
\usepackage{graphicx}
\usepackage{longtable}
\usepackage{wrapfig}
\usepackage{rotating}
\usepackage[normalem]{ulem}
\usepackage{amsmath}
\usepackage{amssymb}
\usepackage{capt-of}
\usepackage{hyperref}
\usepackage[margin=0.5in]{geometry}
\usepackage{graphicx}
\usepackage{fancyhdr}
\pagestyle{fancyplain}
\chead{}
\lhead{}
\rhead{Cho Orgaz \thepage}
\cfoot{CSC664 - Team 11- Project 14}
\lfoot{}
\rfoot{}
\author{Group 11: Andy Cho \& Kevin Islas Orgaz}
\date{\textit{<2022-11-28 Mon>}}
\title{Automatic Event-Based Grouping and Interaction with Personal Multimedia Information}
\hypersetup{
 pdfauthor={Group 11: Andy Cho \& Kevin Islas Orgaz},
 pdftitle={Automatic Event-Based Grouping and Interaction with Personal Multimedia Information},
 pdfkeywords={},
 pdfsubject={Slide Deck for our M3 Presentation},
 pdfcreator={Emacs 28.2 (Org mode 9.5.5)}, 
 pdflang={English}}
\begin{document}


%----------------------------------------------------------------------------------------
%	TITLE PAGE
%----------------------------------------------------------------------------------------

\begin{titlepage} % Suppresses displaying the page number on the title page and the subsequent page counts as page 1
	\newcommand{\HRule}{\rule{\linewidth}{0.5mm}} % Defines a new command for horizontal lines, change thickness here
	
	\center % Centre everything on the page
	
	%------------------------------------------------
	%	Headings
	%------------------------------------------------
	
	\textsc{\LARGE San Francisco State University}\\[1.5cm]
	\textsc{\Large CSC664-01}\\[0.5cm]
	
	
	\textsc{\large Professor Rahul Singh}\\[0.5cm]
	\textsc{\large TA - Fiona Senchyna}\\[0.3cm]
	
	%------------------------------------------------
	%	Title
	%------------------------------------------------
	
	\HRule\\[0.4cm]
	{\huge\bfseries  Automatic Event-Based Grouping and Interaction with Personal Multimedia Information}\\[0.1cm] % Title of your document
	\HRule\\[0.4cm]
	
	%------------------------------------------------
	%	Author(s)
	%------------------------------------------------

	
	\begin{minipage}{0.4\textwidth}
	{\huge\bfseries  Presentation Itinerary}\\[0.5cm] % Title of your document
	\end{minipage}
	
	
	\begin{minipage}{0.4\textwidth}
		\begin{flushleft}
			\textsc{Team 11}\\
			\textsc{Andy Cho}\\
		\end{flushleft}
	\end{minipage}
	~
	\begin{minipage}{0.4\textwidth}
		\begin{flushright}
			\textsc{Project 13}\\
			\textsc{Kevin Islas Orgaz}\\
		\end{flushright}
	\end{minipage}
	
	% If you don't want a supervisor, uncomment the two lines below and comment the code above
	%{\large\textit{Author}}\\
	%John \textsc{Smith} % Your name
	
	%------------------------------------------------
	%	Date
	%------------------------------------------------
	
	\vfill\vfill\vfill % Position the date 3/4 down the remaining page
	

	%------------------------------------------------
	%	Logo
	%------------------------------------------------
	
	\vfill
	 
	{\large\today} % Date, change the \today to a set date if you want to be precise
	\vfill
	%----------------------------------------------------------------------------------------
	
	\vfill % Push the date up 1/4 of the remaining page
	
\end{titlepage}

%----------------------------------------------------------------------------------------

\tableofcontents



\newpage
\section{Dataset and GUI Requirements}
\label{sec:org6068936}
\subsection{Dataset}
\label{sec:org92fcc91}
Our dataset was derived from custom downloaded family event albums from flickr with the original timestamps.
We also included our own photos in order to capture all the raw metadata files can have- both temporally(datetime) and spatially(lat, lon)
We manually tagged our flickr dataset with geo-coordinates for usage in our multimedia system

\subsection{GUI Functional Requirements}
\label{sec:org18914f0}
User Operation List
\begin{center}
\begin{tabular}{ll}
\hline
Operation & I/O\\
\hline
Insert Media & \{MMO\}, event\\
Create Empty Event & \{\}->event\\
Create Event(manual) & \{MMO\}->event\\
Create Events using a cluster algorithm & \{MMO\} -> \{event\}\\
Remove Event & \{MMO\}\\
Attach Event & \{event\},event\\
Detach from super-event & \{event\}, super-event\\
\hline
\end{tabular}
\end{center}



\newpage
\section{Design}
\label{sec:orgb4f7435}
\subsection{Library Gallery: An overview of all multimedia files to be sorted as events}
\label{sec:org68c48d3}
\begin{itemize}
\item mimetypes like 'image/jpeg' categorize all multimedia into distinct categories making it easier to make queries with filetype
\end{itemize}
\subsection{Universal Media Player:}
\label{sec:orgc2e499c}
For a few media types, we use 3rd party libraries, for the majority however, we used standard libraries that pre-exist within our codebase.
\begin{center}
\begin{tabular}{ll}
\hline
Media Types & \\
\hline
application & \\
audio & Various Audio-Processing Libraries for Audio-based event-annotation\\
image & standard html, css, and js libraries\\
message & stl\\
multipart & stl\\
text & stl\\
video & stl\\
\hline
\end{tabular}
\end{center}

\subsection{Timeline: Temporal Clustering based off of different classes of Event-Relations}
\label{sec:org585c2b4}
\textbf{\textbf{LIVE-DEMO}}:
\begin{itemize}
\item Point-Point Relations
\item Point-Interval Relations
\item Interval-Interval Relations
\end{itemize}

\subsection{GPS Event Display:}
\label{sec:orgbb30b01}
\begin{itemize}
\item Estimating Spatial Realms for Events based off of Multimedia Sets
Calculating a set of points and nonconvergent lines
    Methodology, Issues, Solutions
\end{itemize}


\begin{itemize}
\item Powered by GoogleMaps
\begin{itemize}
\item For rapid prototyping purposes, events are clustered into an 'average' set of coordinates and then displayed fed and displayed onto the GoogleMaps world viewer
\end{itemize}
\end{itemize}

\subsection{Media Annotation Tool:}
\label{sec:orgeab4932}
Our Media Annotation Tool is split into 2 parts
\begin{itemize}
\item A simple and interactive way to create CRUD operations on EXIF data for temporal and spatial data.
\end{itemize}
Multimedia-Processing Libraries
\begin{itemize}
\item FacialDetection functionality to tag certain people
\item AudioProcessing to tag important data like meeting subject times
\end{itemize}

\subsection{Event-Detail and Event-Hierarchy Display}
\label{sec:orgcc763d2}
An interactive medium to modify events and their position and status within their event-hierarchy


\newpage
\section{Experimentation: Demonstrating Design Efficacy and Event-Clustering through Experimentation}
\label{sec:org028c8c7}
\subsection{Experimentation}
\label{sec:orgd3257e7}
\subsubsection{Effectiveness of Time-Based Clustering to Define Events}
\label{sec:org2bdb83c}
\subsubsection{Usefulness of the proposed interactions for users in managing information}
\label{sec:orgd88d29e}
\subsubsection{Qualitative and Quantitative results}
\label{sec:orga7d3f3e}
\begin{itemize}
\item Likert Scales, Questionnaires, and Surveys
\item Empirical measurements of access complexity via motions/time
\item Estimate the NASA Task Load Index score for various environments
\end{itemize}
\end{document}